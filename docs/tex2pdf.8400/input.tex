\documentclass[]{article}
\usepackage{lmodern}
\usepackage{amssymb,amsmath}
\usepackage{ifxetex,ifluatex}
\usepackage{fixltx2e} % provides \textsubscript
\ifnum 0\ifxetex 1\fi\ifluatex 1\fi=0 % if pdftex
  \usepackage[T1]{fontenc}
  \usepackage[utf8]{inputenc}
\else % if luatex or xelatex
  \ifxetex
    \usepackage{mathspec}
  \else
    \usepackage{fontspec}
  \fi
  \defaultfontfeatures{Ligatures=TeX,Scale=MatchLowercase}
\fi
% use upquote if available, for straight quotes in verbatim environments
\IfFileExists{upquote.sty}{\usepackage{upquote}}{}
% use microtype if available
\IfFileExists{microtype.sty}{%
\usepackage{microtype}
\UseMicrotypeSet[protrusion]{basicmath} % disable protrusion for tt fonts
}{}
\usepackage[margin=1in]{geometry}
\usepackage{hyperref}
\PassOptionsToPackage{usenames,dvipsnames}{color} % color is loaded by hyperref
\hypersetup{unicode=true,
            colorlinks=true,
            linkcolor=Maroon,
            citecolor=Blue,
            urlcolor=blue,
            breaklinks=true}
\urlstyle{same}  % don't use monospace font for urls
\usepackage{graphicx,grffile}
\makeatletter
\def\maxwidth{\ifdim\Gin@nat@width>\linewidth\linewidth\else\Gin@nat@width\fi}
\def\maxheight{\ifdim\Gin@nat@height>\textheight\textheight\else\Gin@nat@height\fi}
\makeatother
% Scale images if necessary, so that they will not overflow the page
% margins by default, and it is still possible to overwrite the defaults
% using explicit options in \includegraphics[width, height, ...]{}
\setkeys{Gin}{width=\maxwidth,height=\maxheight,keepaspectratio}
\IfFileExists{parskip.sty}{%
\usepackage{parskip}
}{% else
\setlength{\parindent}{0pt}
\setlength{\parskip}{6pt plus 2pt minus 1pt}
}
\setlength{\emergencystretch}{3em}  % prevent overfull lines
\providecommand{\tightlist}{%
  \setlength{\itemsep}{0pt}\setlength{\parskip}{0pt}}
\setcounter{secnumdepth}{0}
% Redefines (sub)paragraphs to behave more like sections
\ifx\paragraph\undefined\else
\let\oldparagraph\paragraph
\renewcommand{\paragraph}[1]{\oldparagraph{#1}\mbox{}}
\fi
\ifx\subparagraph\undefined\else
\let\oldsubparagraph\subparagraph
\renewcommand{\subparagraph}[1]{\oldsubparagraph{#1}\mbox{}}
\fi

%%% Use protect on footnotes to avoid problems with footnotes in titles
\let\rmarkdownfootnote\footnote%
\def\footnote{\protect\rmarkdownfootnote}

%%% Change title format to be more compact
\usepackage{titling}

% Create subtitle command for use in maketitle
\newcommand{\subtitle}[1]{
  \posttitle{
    \begin{center}\large#1\end{center}
    }
}

\setlength{\droptitle}{-2em}
  \title{}
  \pretitle{\vspace{\droptitle}}
  \posttitle{}
  \author{}
  \preauthor{}\postauthor{}
  \date{}
  \predate{}\postdate{}

\usepackage{fancyhdr}
\pagestyle{fancy}
\fancyhead[RE,RO]{Mira Bora}
\fancyhead[LE,LO]{Medialogy, CPH, 2017}
\fancyfoot[CE,CO]{STUDY VERIFICATION TEST}
\fancyfoot[LE,RO]{\thepage}

\begin{document}

\section{Study Verification Test - Student
Report}\label{study-verification-test---student-report}

This is an interpretive report of your responses to the study
verification test. Its purpose is to help you identify your student
profile within specific topics.

The boxplots\footnote{A \emph{boxplot} illustrates the full range of
  variation (from min to max), the likely range of variation called the
  interquartile range (IQR) and a typical value (median). The
  rectangular box denotes the IQR, corresponding to the middle scores of
  the dataset - ranging from the 25th to the 75th percentile. The line
  dividing the box into two parts marks the median. The horizontal lines
  to the left and right of the IQR (whiskers) are 1.5 times as long as
  the width of the IQR. Outliers are defined as data points outside the
  whisker ranges and plotted as black dots.} show how you compare to a
larger sample of first semester Medialogy students from Aalborg and
Copenhagen. Specifically, they indicate the average self-ratings of all
students in seven different topics (see Figure 1) and the self-reported
study hours per week (see Figure 2). Your self-reported value for each
topic is indicated with red dots. A value less than 0.5 means that you
rated yourself lower than the average student. The percentiles indicate
the percentage of students whose scores are equal to or less than yours.
Based on these results we have created specific recommendations for you
to get more comfortable in the Medialogy study environment.

As the report is based on the questionnaire information alone, it may
give only a rough indication of your true attitudes. Your advisor or
student counselor will help you understand your scores and find the
services you desire.

\includegraphics{C:/Users/BiancaClavio/Documents/SVN/01Projects/SSP/handouts/SSP_CPH_Individual-student-feedback_mbora17_files/figure-latex/unnamed-chunk-4-1.pdf}

\includegraphics{C:/Users/BiancaClavio/Documents/SVN/01Projects/SSP/handouts/SSP_CPH_Individual-student-feedback_mbora17_files/figure-latex/unnamed-chunk-5-1.pdf}

\pagebreak

\subsection{Specific Recommondations}\label{specific-recommondations}

In this section you will receive a more detailed explanation of your
results. The purpose of this information is to help you develop your
skills and get the most from your university experience. Take a balanced
approach to reviewing and utilizing this information. Do not assume that
each statement is perfectly accurate just because it is printed in a
formal manner; some statements may not fit you well. However, do not
dismiss a statement just because it points to a challenge.

Keep an open mind as you consider each statement. When it seems
accurate, give serious thought to any suggestions that accompany the
statement. If the statement is puzzling, discuss it with someone who can
help you interpret it. Approaching the information in this way can be
very helpful.

\paragraph{Social Support for
Studying}\label{social-support-for-studying}

Studying is a long-term endeavour and there will be times of frustration
and doubt. Your self-reported social support for studying placed you in
the 42nd percentile, and your responses suggest that you enrolled for a
university degree in general and Medialogy at AAU specifically without
having received a large amount of encouragement from friends, family, or
other sources. During your education it can help to have a social
network that understands that times of frustration can be part of
pursuing a higher education degree. Your social network can support you
in times of hardship, doubt, and low morale. It is good to hear that you
have already made some friends at AAU. Making friends with fellow
students at university can be a valuable source to rely on throughout
the education. Taking active part in student unions, being a volunteer
at events, or other types of student environment activities can help
you. It will broaden your social reference beyond your own semester or
education. Breaking the barrier to seek social network outside your own
semester can seem daunting, but can become a significantly positive part
of your years as university student. If in doubt about how to initiate
or engage, consider having a chat with your student counselor.

\paragraph{High School Habits}\label{high-school-habits}

In high school, the teacher often has the responsibility of giving
homework, communicating learning material, recording attendance in
class, ensure student progress, and help the students when required.
Your self-reported high school habits placed you in the 100th
percentile, which suggests that you have above average study habits.
However, remember that going from high school to univesity involves many
changes that you need to adjust to. As a student at university, you have
the responsibility for what you learn. Your lecturers will often have
more focus on academic content than on pedagogy, and weak study habits
can therefore set you back in your learning progress.

\paragraph{Study Habits}\label{study-habits}

Weak study habits are the single greatest cause of academic problems at
university. Your self-reported study habits placed you in the 98th
percentile, suggesting that you are disciplined and know quite well what
it takes to study at university. Although your self-reported study
habits are above average, you still want to put effort into this area,
as you will experience many changes since high school. Develop a clear
daily routine in which you set aside certain periods of time to study.
Learn to focus your attention and to pace yourself. Other useful
techniques include previewing, underlining, note-taking, and reviewing.
In case you want to further improve your study habits have a look
\href{tinyurl.com/AAUstudyPlanning}{here} (tinyurl.com/AAUstudyPlanning)
and find study related exercises
\href{tinyurl.com/AAUstudySkillsExercises}{here}
(tinyurl.com/AAUstudySkillsExercises).

\paragraph{Grit}\label{grit}

Talent without hard work rarely amounts to anything ambitious. Making
the effort to stay with a problem (or challenge) for long enough,
increases your chances of cracking it and mastering new skills. This
requires time and dedication, and is known as grit. Students with high
self-reported values in grit, are less likely to drop out and fail exams
than those with lower scores. Your self-reported grit places you in the
85th percentile, and your perseverance and dedication to solving
problems are therefore above average. Previous Medialogy cohorts have
shown that even students with poor high school grades will make it
through the education if they persist and invest the time and effort.
Remember this next time you encounter a difficult problem and you feel
like giving up.

\paragraph{Growth mindset}\label{growth-mindset}

Seeing your intelligence as something that you can actively influence
and grow is referred to as having a growth mindset. Students with high
self-reported values in growth mindset are less likely to drop out and
fail exams than those with lower scores. Your self-reported
understanding of intelligence placed you in the 5th percentile, meaning
that you see intelligence as something rather static and outside of your
control to affect. Instead of seeing setbacks as a manifestation of
inadequacy, lack of intelligence, or lack of talent, consider them an
opportunity to learn and grow your mindset, to overcome academic
challenges in your study life. Develop a positive learning attitude,
where challenge is not unwanted, but a logical part of the growth
journey. Such challenges include lower than expected outcomes such as
failed assignments, exams or low grades. We all have talents and
weaknesses. We naturally need to work harder on our personal weaknesses,
but we are still capable.

\paragraph{Study and Work}\label{study-and-work}

Studying requires a lot of time and dedication in order to succeed.
Being intelligent and having talent can help but does not replace the
need for dedicating time and effort to studying. The ECTS system assumes
that you spend 45 hours a week on your education. You reported using 38
hours weekly for studying, which indicates that you are using less than
the recommended amount of time for studying. Meeting these demands is
difficult over long-term with too many other obligations. You reported
using 30 hours on study related work, and 10 hours on non-study related
work. Should you not able to dedicate around 45 hours to study each
week, you should not despair when you fail exams. You simply did not
have the time resources to succeed and studying might take longer than
expected. However, if you mainly rely on SU this provides a clear time
frame within which you need to finish your education. You should
therefore carefully review your commitments and other activities that
you need or want to dedicate time to vis-à-vis the study demands.

\paragraph{Understanding of Medialogy}\label{understanding-of-medialogy}

Choosing a suitable education can be difficult, and students should
reflect on their choice of education, especially, in the first
semester(s). Your understanding about what you will learn in the
Medialogy programme is in the 5th percentile, meaning that your
understanding of Medialogy is rather different from what the study plan
indicates. Medialogy has a strong technical foundation but offers many
other skills and competencies that give you opportunities to design
novel technologies and this usually involves programming. Contrary to
what you indicated in the study verification questionnaire Medialogy
students learn only little about: manual content creation, work with
aesthetics, professional moving making, working as a rigging artist, and
using adobe software Although you have group responsibilities related to
you project, the group is not responsible of the individual learning
goals, defined in the study plan. Previous students with different
expectations either learned to appreciate the content, skills, and
opportunities of a Medialogy degree or changed to other programmes or
educations. Attending the Med Awards event (usually in November) and
seeing other student projects can give you an idea of what you will be
able to do if you apply yourself to what Medialogy can offer. Contacting
older students in the Medialogy facebook group, study counselors, and
going to the study cafe can also provide you information about the
education and what it means to be a Medialogy student. You can find
information about Medialogy and student testimonials
\href{tinyurl.com/AAUaboutMedialogy}{here}
(tinyurl.com/AAUaboutMedialogy).


\end{document}
