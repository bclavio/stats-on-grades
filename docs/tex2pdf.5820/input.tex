\documentclass[]{article}
\usepackage{lmodern}
\usepackage{amssymb,amsmath}
\usepackage{ifxetex,ifluatex}
\usepackage{fixltx2e} % provides \textsubscript
\ifnum 0\ifxetex 1\fi\ifluatex 1\fi=0 % if pdftex
  \usepackage[T1]{fontenc}
  \usepackage[utf8]{inputenc}
\else % if luatex or xelatex
  \ifxetex
    \usepackage{mathspec}
  \else
    \usepackage{fontspec}
  \fi
  \defaultfontfeatures{Ligatures=TeX,Scale=MatchLowercase}
\fi
% use upquote if available, for straight quotes in verbatim environments
\IfFileExists{upquote.sty}{\usepackage{upquote}}{}
% use microtype if available
\IfFileExists{microtype.sty}{%
\usepackage{microtype}
\UseMicrotypeSet[protrusion]{basicmath} % disable protrusion for tt fonts
}{}
\usepackage[margin=1in]{geometry}
\usepackage{hyperref}
\PassOptionsToPackage{usenames,dvipsnames}{color} % color is loaded by hyperref
\hypersetup{unicode=true,
            colorlinks=true,
            linkcolor=Maroon,
            citecolor=Blue,
            urlcolor=blue,
            breaklinks=true}
\urlstyle{same}  % don't use monospace font for urls
\usepackage{longtable,booktabs}
\usepackage{graphicx,grffile}
\makeatletter
\def\maxwidth{\ifdim\Gin@nat@width>\linewidth\linewidth\else\Gin@nat@width\fi}
\def\maxheight{\ifdim\Gin@nat@height>\textheight\textheight\else\Gin@nat@height\fi}
\makeatother
% Scale images if necessary, so that they will not overflow the page
% margins by default, and it is still possible to overwrite the defaults
% using explicit options in \includegraphics[width, height, ...]{}
\setkeys{Gin}{width=\maxwidth,height=\maxheight,keepaspectratio}
\IfFileExists{parskip.sty}{%
\usepackage{parskip}
}{% else
\setlength{\parindent}{0pt}
\setlength{\parskip}{6pt plus 2pt minus 1pt}
}
\setlength{\emergencystretch}{3em}  % prevent overfull lines
\providecommand{\tightlist}{%
  \setlength{\itemsep}{0pt}\setlength{\parskip}{0pt}}
\setcounter{secnumdepth}{0}
% Redefines (sub)paragraphs to behave more like sections
\ifx\paragraph\undefined\else
\let\oldparagraph\paragraph
\renewcommand{\paragraph}[1]{\oldparagraph{#1}\mbox{}}
\fi
\ifx\subparagraph\undefined\else
\let\oldsubparagraph\subparagraph
\renewcommand{\subparagraph}[1]{\oldsubparagraph{#1}\mbox{}}
\fi

%%% Use protect on footnotes to avoid problems with footnotes in titles
\let\rmarkdownfootnote\footnote%
\def\footnote{\protect\rmarkdownfootnote}

%%% Change title format to be more compact
\usepackage{titling}

% Create subtitle command for use in maketitle
\newcommand{\subtitle}[1]{
  \posttitle{
    \begin{center}\large#1\end{center}
    }
}

\setlength{\droptitle}{-2em}
  \title{}
  \pretitle{\vspace{\droptitle}}
  \posttitle{}
  \author{}
  \preauthor{}\postauthor{}
  \date{}
  \predate{}\postdate{}

\usepackage{fancyhdr}
\pagestyle{fancy}
\fancyhead[RE,RO]{Anonymous student}
\fancyhead[LE,LO]{Medialogy, AAL, 2017}
\fancyfoot[CE,CO]{INTRODUCTION TO PROGRAMMING}
\fancyfoot[LE,RO]{\thepage}

\begin{document}

\section{Midterm Exam - Student
Report}\label{midterm-exam---student-report}

This is a report of your acedemic performance in the introduction to
programming course (GPRO), including your scores in the midterm exam
(MT) and the self-assessment quizzes (SA). Our data analysis of
Medialogy students in Aalborg shows that students taking the SA's have
higher scores in the MT, regardless of prior programming experience. We
hope that you will benefit from this report by identifying your current
programming progress.

The boxplots\footnote{A \emph{boxplot} illustrates the full range of
  variation (from min to max), the likely range of variation called the
  interquartile range (IQR) and a typical value (median). The
  rectangular box denotes the IQR, corresponding to the middle scores of
  the dataset - ranging from the 25th to the 75th percentile. The line
  dividing the box into two parts marks the median. The horizontal lines
  to the left and right of the IQR (whiskers) are 1.5 times as long as
  the width of the IQR. Outliers are defined as data points outside the
  whisker ranges and plotted as black dots.} show how you compare to a
larger sample of first semester Medialogy students from Aalborg and
Copenhagen (see Figure 1). Specifically, they indicate the average score
of all students in five topics, based on results from the SA and the MT.
The SA topics refer to the first self-assessment quizzes on Moodle and
their corresponding lectures, i.e.~course introduction, variables and
math, branching, looping, and functions. The same five topics constitute
the midterm exam questions (Q). Your score in the MT is indicated with
red dots, and your score in the SA is indicated with blue dots. The
scores are normalized on a scale from 0 to 1, indicating the lowest and
highest possible scores for each topic. A score above 0.5 means that you
received more than half of the points denoted for the particular topic.
The percentiles indicate the percentage of students whose scores are
equal to or less than yours.

\includegraphics{C:/Users/BiancaClavio/Documents/SVN/01Projects/GPRO/handoutsGPROSA/GPRO-MT_AAL_Individual-StudentFeedback_aalota17_files/figure-latex/unnamed-chunk-5-1.pdf}

\begin{center}\rule{0.5\linewidth}{\linethickness}\end{center}

\subsection{Programming progress - from SA to
MT}\label{programming-progress---from-sa-to-mt}

Table 1 describes characterics of your programming experience, SA usage,
and MT score. The SA data consists of the first seven SA's, and it was
retrieved one day before the MT. Thus, with this data we analyse if the
SA usage before the MT is related to MT scores. For this purpose, we
look at the percentage of completed SA's, the average score of all seven
SA's (assigning zero to incompleted SA's), and the average score of
completed SA's.

In September you reported having novice programming experience in the
study verification test (SSP). You reached a score of 42.08 in the MT,
showing that you need to study and practise programming to be on-track
with the course. Some students with no and little prior programming
experience managed to achieve high scores in the MT (see Table 2), and
with more effort and time you can do the same. Students, with similar
prior experienced as you, obtained high MT scores if they had completed
the SA's and achieved a high score in these. Hence, you can use this
tool to assess your learning progress. Other tools and activities also
have influence on learning, and you should utilize all accessible aids.
For example, asking for help from the teacher and TA's not only assist
you in completing the mandatory exercises, but it can also help you in
understanding what is essential for you to learn for the course. In
November 2017, you had the advantage of such opportunaty in an
individual tutoring session, from which you could get help with anything
in programming. At this time of the year, with no lectures, you will
need to seek support elsewhere, such as your peers. Teaming up with
peers and explaining to each other the course content can help you reach
new understandings, other methods to apply programming, and which topics
of the course you need to review.

\paragraph{Table 1: Statistics of programming experience, SA usage, and
MT
score.}\label{table-1-statistics-of-programming-experience-sa-usage-and-mt-score.}

\begin{longtable}[]{@{}cccccc@{}}
\toprule
\begin{minipage}[b]{0.09\columnwidth}\centering\strut
Prior exp.\strut
\end{minipage} & \begin{minipage}[b]{0.17\columnwidth}\centering\strut
Perc. of completed SA's\strut
\end{minipage} & \begin{minipage}[b]{0.14\columnwidth}\centering\strut
Avg. score of SA's\strut
\end{minipage} & \begin{minipage}[b]{0.20\columnwidth}\centering\strut
Avg. score of completed SA's\strut
\end{minipage} & \begin{minipage}[b]{0.11\columnwidth}\centering\strut
MT score\strut
\end{minipage} & \begin{minipage}[b]{0.11\columnwidth}\centering\strut
Tutoring\strut
\end{minipage}\tabularnewline
\midrule
\endhead
\begin{minipage}[t]{0.09\columnwidth}\centering\strut
novice\strut
\end{minipage} & \begin{minipage}[t]{0.17\columnwidth}\centering\strut
100\strut
\end{minipage} & \begin{minipage}[t]{0.14\columnwidth}\centering\strut
94.55\strut
\end{minipage} & \begin{minipage}[t]{0.20\columnwidth}\centering\strut
94.55\strut
\end{minipage} & \begin{minipage}[t]{0.11\columnwidth}\centering\strut
42.08\strut
\end{minipage} & \begin{minipage}[t]{0.11\columnwidth}\centering\strut
not taken\strut
\end{minipage}\tabularnewline
\bottomrule
\end{longtable}

Table 2 shows the minimum (min), maximum (max), and average (avg) of the
midterm exam scores based on prior programming experience, retrieved
from the study verification test (SSP). The descriptive statistics
consist of data from Medalogy students in Aalborg and Copenhagen, of
which the number of students in each experience level is indicated with
N. As you may have heard or experienced, hands-on practice is the key to
learning and improving your programming skills. Actively participating
and completing all lecture exercises is a reason why some of your peers
across prior programming experience achieved a high score.

\paragraph{Table 2: MT scores based on prior programming
experience.}\label{table-2-mt-scores-based-on-prior-programming-experience.}

\begin{longtable}[]{@{}lrrrr@{}}
\toprule
Prior programming experience & min & max & avg & N\tabularnewline
\midrule
\endhead
beginner (no experience) & 32.16 & 91.02 & 61.89 & 65\tabularnewline
novice (small projects less than 900 lines) & 26.55 & 99.55 & 66.22 &
85\tabularnewline
intermediate (projects btw. 900-40.000 lines) & 38.71 & 99.55 & 71.66 &
22\tabularnewline
expert (projects more than 40.000 lines) & 57.27 & 76.59 & 65.95 &
3\tabularnewline
\bottomrule
\end{longtable}


\end{document}
